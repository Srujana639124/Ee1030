%iffalse
\let\negmedspace\undefined
\let\negthickspace\undefined
\documentclass[journal,12pt,twocolumn]{IEEEtran}
\usepackage{cite}
\usepackage{amsmath,amssymb,amsfonts,amsthm}
\usepackage{algorithmic}
\usepackage{graphicx}
\usepackage{textcomp}
\usepackage{xcolor}
\usepackage{txfonts}
\usepackage{listings}
\usepackage{enumitem}
\usepackage{mathtools}
\usepackage{gensymb}
\usepackage{comment}
\usepackage[breaklinks=true]{hyperref}
\usepackage{tkz-euclide} 
\usepackage{listings}
\usepackage{gvv}                                        
%\def\inputGnumericTable{}                                 
\usepackage[latin1]{inputenc}                                
\usepackage{color}                                            
\usepackage{array}                                            
\usepackage{longtable}                                       
\usepackage{calc}                                             
\usepackage{multirow}                                         
\usepackage{hhline}                                           
\usepackage{ifthen}                                           
\usepackage{lscape}
\usepackage{tabularx}
\usepackage{array}
\usepackage{float}
\usepackage{multicol}


\newtheorem{theorem}{Theorem}[section]
\newtheorem{problem}{Problem}
\newtheorem{proposition}{Proposition}[section]
\newtheorem{lemma}{Lemma}[section]
\newtheorem{corollary}[theorem]{Corollary}
\newtheorem{example}{Example}[section]
\newtheorem{definition}[problem]{Definition}
\newcommand{\BEQA}{\begin{eqnarray}}
\newcommand{\EEQA}{\end{eqnarray}}
\newcommand{\define}{\stackrel{\triangle}{=}}
\theoremstyle{remark}
\newtheorem{rem}{Remark}

% Marks the beginning of the document
\begin{document}
\bibliographystyle{IEEEtran}
\vspace{3cm}

\title{\textbf{FUNCTIONS}}
\author{\textbf{SRUJANA-EE24BTECH11042}}


\maketitle
\newpage
\bigskip

\renewcommand{\thefigure}{\theenumi}
\renewcommand{\thetable}{\theenmui}
\begin{large}
	\textbf{SECTION-A(D)}
\end{large}
\begin{enumerate}[start=2]
\item Let $g\brak{x}$ be a function defined on $\sbrak{-1,1}$.if the area of the equilateral triangle with two of its vertices at $\brak{0,0}$ and $\sbrak{x,g\brak{x}}$ is $\frac{\sqrt{3}}{4}$, then the function $g\brak{x}$ is
\hfill(1989-2 Marks)
\begin{multicols}{2}
\begin{enumerate} 
\item $g\brak{x}=\pm\sqrt{1-x^2}$ 
\item $g\brak{x}=\sqrt{1-x^2}$
\item $g\brak{x}=-\sqrt{1-x^2}$
\item $g\brak{x}=\sqrt{1+x^2}$
\end{enumerate}
\end{multicols}
\item If $f\brak{x} = \cos\sbrak{\pi^2}x + cos\sbrak{-\pi^2}x$, where
\sbrak{x} stands for the greatest integer function , then  

\hfill(1991-2Marks)
\begin{multicols}{2}
\begin{enumerate} 
\item $f\brak{\frac{\pi}{2}}=-1$
\item $f\brak{\pi}=1$
\item $f\brak{-\pi}=0$
\item $f\brak{\frac{\pi}{4}}=1$
\end{enumerate}
\end{multicols}
\item If $f\brak{x}=3x-5$,then $f^{-1}\brak{x}$
\hfill(1998-2Marks)
\begin{enumerate} 
\item is given by $\frac{1}{3x-5}$
\item is given by $\frac{x+5}{3}$
\item does not exist because $f$ is not one$-$one
\item does not exist because $f$ is not onto 
\end{enumerate}
\item If $g\brak{f\brak{x}}=\abs\sin{x}$ and $f\brak{g\brak{x}}=\brak{\sin{\sqrt{x}}}^2$,then
\hfill(1998-2Marks)
\begin{enumerate} 
\item $f\brak{x}=\sin{x}^2,g\brak{x}=\sqrt{x}$
\item $f\brak{x}=\sin{x},g\brak{x}=\abs x$
\item $f\brak{x}=x^2,g\brak{x}=\sin{\sqrt{x}}$
\item $f$ and $g$ cannot be determined
\end{enumerate}
\item Let $f:\brak{0,1}\rightarrow{R}$ be defined by $f\brak{x}=\frac{b-x}{1-bx}$,where b is a constant such that $0<b<1$.Then

\begin{enumerate}
\item $f$ is not invertible  on $\brak{0,1}$
\item$f\neq{f^{-1}}$ on $(0,1)$ and $f^{1}(b)=\frac{1}{f^{1}\brak{0}}$
\item(c)$f=f^{-1}$ on $\brak{0,1}$ and $f^{1}\brak{b}=\frac{1}{f^{1}\brak{0}}$
\item$f^{-1} $is differentiable $(0,1)$
\end{enumerate}
\item Let $f:\brak{-1,1}\rightarrow{IR}$ be  such that $f\brak{\cos{4\theta}} =\frac{2}{2-\sec^2{\theta}}$ for $\theta \in \brak{0,\frac{\pi}{4}}\cup\brak{\frac{\pi}{4},\frac{\pi}{2}}$. Then the value(s) of $f\brak{\frac{1}{3}}$is are
\begin{multicols}{2}
\begin{enumerate} 
\item $1-\sqrt{\frac{3}{2}}$  
\item $1+\sqrt{\frac{3}{2}}$
\item $1-\sqrt{\frac{2}{3}}$   
\item $1+\sqrt{\frac{2}{3}}$
\end{enumerate}
\end{multicols}
\item The function $f\brak{x}=2\abs x+\abs {x+2}-2\abs x$ has local minimum or local maximum at x$=$

\hfill(JEE Adv.2013)
\begin{multicols}{4}
\begin{enumerate}
\item $-$2
\item $\frac{-2}{3}$
\item 2
\item$\frac{2}{3}$
\end{enumerate}
\end{multicols}
\item Let $f:\brak{\frac{-\pi}{2},\frac{\pi}{2}}\rightarrow{R}$ be given by $f\brak{x}=\brak{log\brak{\sec{x}+tan{x}}}^3$ Then
\hfill(JEE Adv.2014)
\begin{enumerate} 
\item$f\brak{x}$ is an odd function
\item$f\brak{x}$ is one-one function
\item$f\brak{x}$ is an onto function
\item$f\brak{x}$ is an even function
\end{enumerate}
\item Let $a \in R$ and let $f:R\rightarrow{R}$ be given by $f\brak{x}=x^5-5x+a$.Then
\hfill(JEE Adv.2014)
\begin{enumerate} 
\item $f\brak{x}$ has three real roots if $a>4$
\item $f\brak{x}$ has only real root if $a>4$
\item $f\brak{x}$ has three real roots if $a<-4$
\item $f\brak{x}$ has three real roots if $-4<a<4$
\end{enumerate}
\item Let $f\brak{x}=\sin\brak{\frac{\pi}{6}\brak{\frac{\pi}{2}\sin{x}}}$ for all $x \in R $ and $g\brak{x}=\frac{\pi}{2}\sin{x}$ for all $x \in R$. Let $\brak{fog}\brak{x}$ denote $f\brak{g\brak{x}}$ and $\brak{gof}\brak{x}$ denote $g\brak{f\brak{x}}.$ Then which of the following are true $?$

\hfill(JEE Adv 2015)
\begin{enumerate} 
\item Range of $f$ is $\sbrak{\frac{-1}{2},\frac{1}{2}}$
\item Range of $fog$ is $\sbrak{\frac{-1}{2},\frac{1}{2}}$
\item $\lim_{x\rightarrow{0}}\frac{f(x)}{g(x)}=\frac{\pi}{6}$
\item There is an $x \in R$ such that $\brak{gof}\brak{x}=1$
\end{enumerate}
\end{enumerate}
\begin{large}
      \textbf{SECTION-A(E)}
\end{large}
\begin{enumerate}
\item Find the domain and the range of the function $f\brak{x}+\frac{x^2}{1+x^2}$.Is the function one one? 
\hfill(1978)
\item Draw the graph of $y=\abs x^{\frac{1}{2}}$ for $-1\le x \le1.$

\hfill(1978)
\item If $f\brak{x}=x^9-6x^8-2x^7+12x^6+x^4-7x^3+6x^2+x-3$ find $f\brak{x}$
\hfill(1979)
\item Consider the following relations in the set of real numbers R.

$R=\{\brak{x,y}:x\in R,y\in R,x^2+y^2\le25\}$
$R^1=\{\brak{x,y}:x\in R,y\in R,y\ge\frac{4}{9}x^2\}$

Find the domain and the range of$ R\cap R^1.$ Is the relation$ R\cap R^1$ a function? 
\hfill(1979)
\item Let $A$ and $B$ be two sets each with a finite number of elements. Assume that there is an injective mapping from $A$ to $B$ and that there is an injective mapping from $B$ to $A$.Prove that there is a bijective mapping from $A$ to $B$.
\hfill(1981-2Marks)
\end{enumerate}
\end{document}
