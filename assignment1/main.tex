%iffalse
\let\negmedspace\undefined
\let\negthickspace\undefined
\documentclass[journal,12pt,twocolumn]{IEEEtran}
\usepackage{cite}
\usepackage{amsmath,amssymb,amsfonts,amsthm}
\usepackage{algorithmic}
\usepackage{graphicx}
\usepackage{textcomp}
\usepackage{xcolor}
\usepackage{txfonts}
\usepackage{listings}
\usepackage{enumitem}
\usepackage{mathtools}
\usepackage{gensymb}
\usepackage{comment}
\usepackage[breaklinks=true]{hyperref}
\usepackage{tkz-euclide} 
\usepackage{listings}
\usepackage{gvv}                                        
%\def\inputGnumericTable{}                                 
\usepackage[latin1]{inputenc}                                
\usepackage{color}                                            
\usepackage{array}                                            
\usepackage{longtable}                                       
\usepackage{calc}                                             
\usepackage{multirow}                                         
\usepackage{hhline}                                           
\usepackage{ifthen}                                           
\usepackage{lscape}
\usepackage{tabularx}
\usepackage{array}
\usepackage{float}


\newtheorem{theorem}{Theorem}[section]
\newtheorem{problem}{Problem}
\newtheorem{proposition}{Proposition}[section]
\newtheorem{lemma}{Lemma}[section]
\newtheorem{corollary}[theorem]{Corollary}
\newtheorem{example}{Example}[section]
\newtheorem{definition}[problem]{Definition}
\newcommand{\BEQA}{\begin{eqnarray}}
\newcommand{\EEQA}{\end{eqnarray}}
\newcommand{\define}{\stackrel{\triangle}{=}}
\theoremstyle{remark}
\newtheorem{rem}{Remark}

% Marks the beginning of the document
\begin{document}
\bibliographystyle{IEEEtran}
\vspace{3cm}

\title{\textbf{FUNCTIONS}}
\author{\textbf{SRUJANA\\EE24BTECH11042\\SECTION-A(D)\\MCQs with One or More than One Correct Answer(2-11)}}


\maketitle
\newpage
\bigskip

\renewcommand{\thefigure}{\theenumi}
\renewcommand{\thetable}{\theenumi}

\begin{enumerate}

\item[\textbf{2}]. Let $g\brak{x}$ be a function defined on [-1,1].if the area of the 
   equilateral triangle with two of its vertices at (0,0) and [x,g\brak{x}] is $\frac{\sqrt{3}}{4}$, then the function $g\brak{x}$ is
   
   \hfill\textbf{(1989-2 Marks)}\\
   
(a) $g\brak{x}$ = $\pm\sqrt{1-x^2}$\hspace{1cm} (b) $g\brak{x} = \sqrt{1-x^2}$

(c) $g\brak{x}$  = $-\sqrt{1-x^2}$\hspace{1cm}(c) $g\brak{x}=\sqrt{1+x^2}$\\

\item[\textbf{3}]. If $f\brak{x} = \cos[\pi^2]x + cos[-\pi^2]x$, where [x] stands for the greatest integer function , then  

\hfill\textbf{(1991-2Marks)}

(a) $f(\frac{\pi}{2})=-1$\hspace{1cm}(b) $f(\pi)=1$\\
(C) $f(-\pi)=0$\hspace{1cm}(d) $f(\frac{\pi}{4})=1$\\

\item[\textbf{4}]. If $f(x)=3x-5$,then $f^{-1}(x)$

\hfill\textbf{(1998-2Marks)}

(a) is given by $\frac{1}{3x-5}$\\

(b) is given by $\frac{x+5}{3}$

(c) does not exist because $f$ is not one$-$one\\
(d) does not exist because $f$ is not onto\\

\item[\textbf{5}].If $g(f(x))=|{\sin{x}}| and f(g(x))=(\sin{\sqrt{x}})^2$,then

\hfill\textbf{(1998-2Marks)}

(a) $f(x)=\sin{x}^2,g(x)=\sqrt{x}$\\
(b) $f(x)=\sin{x},g(x)=|x|$

(c) $f(x)=x^2,g(x)=\sin{\sqrt{x}}$\\
(d) $f$ and $g$ cannot be determined\\

\item[\textbf{6}]. Let $f:(0,1)\rightarrow{R}$ be defined by $f(x)=\frac{b-x}{1-bx}$,where b is a constant such that $0<b<1$.Then\\
(a) $f$ is not invertible  on $(0,1)$\\
(b)$f\neq{f^{-1}}$ on $(0,1)$ and $f^{1}(b)=\frac{1}{f^{1}(0)}$\\
(c)$f=f^{-1}$ on (0,1) and $f^{1}(b)=\frac{1}{f^{1}(0)}$\\
(d)$f^{-1} $is differentiable $(0,1)$\\
\newpage
\item[\textbf{7}]. Let $f:\brak{-1,1}\rightarrow{IR}$ be  such that $f\brak{\cos{4\theta}} =\frac{2}{2-\sec^2{\theta}}$ for $\theta \in \brak{0,\frac{\pi}{4}}\cup\brak{\frac{\pi}{4},\frac{\pi}{2}}$. Then the value(s) of $f\brak{\frac{1}{3}}$is are\\
(a) $1-\sqrt{\frac{3}{2}}$  \hspace{1cm} (b)$1+\sqrt{\frac{3}{2}}$\\
(c) $1-\sqrt{\frac{2}{3}}$   \hspace{1cm} (d)$1+\sqrt{\frac{2}{3}}$\\

\item[\textbf{8}]. The function $f(x)=2|x|+|x+2|-2|x|$ has local minimum or local maximum at x$=$

\hfill\textbf{(JEE Adv.2013)}

(a)-2\hspace{1cm}
(b)$\frac{-2}{3}$\hspace{1cm}
(c)2\hspace{1cm}
(d)$\frac{2}{3}$\\

\item[\textbf{9}].Let $f:\brak{\frac{-\pi}{2},\frac{\pi}{2}}\rightarrow{R}$ be given by $f\brak{x}=\brak{log\brak{\sec{x}+tan{x}}}^3$ Then

\hfill\textbf{(JEE Adv.2014)}

(a)$f(x)$ is an odd function

(b)$f(x)$ is one-one function

(c)$f(x)$ is an onto function

(d)$f(x)$ is an even function\\

\item[\textbf{10}]. Let $a \in R$ and let $f:R\rightarrow{R}$ be given by $f\brak{x}=x^5-5x+a$.Then

\hfill\textbf{(JEE Adv.2014)}

(a) $f(x)$ has three real roots if $a>4$\\
(b) $f(x)$ has only real root if $a>4$\\
(c) $f(x)$ has three real roots if $a<-4$\\
(d) $f(x)$ has three real roots if $-4<a<4$\\

\item[\textbf{11}]. Let $f\brak{x}=\sin(\frac{\pi}{6}(\frac{\pi}{2}\sin{x}))$ for all $x \in R $ and $g(x)=\frac{\pi}{2}\sin{x}$ for all $x \in R$. Let $\brak{fog}\brak{x}$ denote $f\brak{g\brak{x}}$ and $\brak{gof}\brak{x}$ denote g\brak{f\brak{x}}. Then which of the following are true $?$

\hfill\textbf{(JEE Adv 2015)}\\
(a) Range of $f$ is $[\frac{-1}{2},\frac{1}{2}]$\\
(b) Range of $fog$ is $[\frac{-1}{2},\frac{1}{2}]$\\
(c) $\lim_{x\rightarrow{0}}\frac{f(x)}{g(x)}=\frac{\pi}{6}$\\
(d) There is an $x \in R$ such that $\brak{gof}\brak{x}=1$\\
\begin{center}
    \item 
      \textbf{SECTION-A(E)\\Subjective Problems(1-5)}\\
      
      
\end{center}

\
\item[\textbf{1}]. Find the domain and the range of the function $f(x)+\frac{x^2}{1+x^2}$.Is the function one one? 
    \hfill\textbf{(1978)}\\
     \item[\textbf{2}]. Draw the graph of $y=|x|^{\frac{1}{2}}$ for $1\le x \le1.$
     
     \hfill\textbf{(1978)}\\
     \item[\textbf{3}].If $f(X)=x^9-6x^8-2x^7+12x^6+x^4-7x^3+6x^2+x-3$ find $f(x)$
     \hfill\textbf{(1979)}\\

     \item[\textbf{4}].Consider the following relations in the set of real numbers R.\\
      $R=\{(x,y):x\in R,y\in R,x^2+y^2\le25\}$\\
      $R^1=\{(x,y):x\in R,y\in R,y\ge\frac{4}{9}x^2\}$\\
      Find the domain and the range of$ R\cap R^1.$ Is the relation$ R\cap R^1$ a function? 
      \hfill\textbf{(1979)}\\

      \item[\textbf{5}].Let A and B be two sets each with a finite number of elements. Assume that there is an injective mapping from A to B and that there is an injective mapping from B to A.Prove that there is a bijective mapping from A to B.

 \hfill\textbf{(1981-2Marks)}\\
 \end{enumerate}
\end{document}
